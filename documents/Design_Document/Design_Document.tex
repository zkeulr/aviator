\documentclass[letterpaper, 11pt]{article}
\usepackage{cite,enumitem,listings, comment}
\usepackage[export]{adjustbox}
\usepackage{amsmath,amssymb,amsfonts}
\usepackage{algorithmic}
\usepackage{graphicx}
\usepackage{textcomp}
\usepackage{graphicx}
\usepackage{geometry}
\usepackage{tabularx}
\usepackage{xcolor}
\usepackage{soul}
\usepackage{chngcntr}
\usepackage{url}
\usepackage{hyperref}
\usepackage{graphicx}
\usepackage{float}
\usepackage{placeins}
\usepackage{caption}
\counterwithin{figure}{subsection}

\begin{document}

%%%%%%%%%%%%%%%% Document Starts %%%%%%%%%%%%%%%%%%%%%

% Title page 
\title{Aviator Design Document}
\author{David Thoe, Joshua Kim, Zeke Ulrich, Juan Vargas}
\date{\today}
\maketitle

\begin{center}
    GTA: Zixiao Ma \\
    Professor: Ryan Beasley
\end{center}

% Sketch goes here
\begin{figure}[h]
    \centering
    \includegraphics[width=12cm,scale=1]{images/white.png}
    \caption{[Caption]}
\end{figure}

\clearpage
\tableofcontents

\clearpage
\listoffigures

\clearpage
\listoftables

% Revision Log Page
\clearpage
\section*{Revision Log}
% For every update, copy a whole LATEX code line and adjust 
% accordingly. In theory, update every time you make major 
% changes in the document. The \& symbol serves to separate 
% items in the table. If you are new to this, align the \& 
% so that it looks more like a table in the code. 
\begin{table}[h]
    \begin{tabularx}{\textwidth}{|l|c|X|} % 
        \hline
        Date      & Revision & Changes         \\ \hline
        5/3/2024  & v0.1     & Initial Release \\ \hline
        9/18/2025 & v1.0     & First Draft     \\ \hline
    \end{tabularx}
    \caption{Revision Log}
\end{table}

% Glossary Page
\clearpage
\section*{Glossary} % Document key words and acronyms here. 
\begin{itemize}
    \item \textbf{API} Application Programming Interface.
\end{itemize}

\clearpage
\section{Introduction}
\subsection{Executive Description}
[Type here. \textbf{DD1+}]
\subsection{User Story}
[Type here. \textbf{DD1+}]
\clearpage
\section{Design Requirements}
\subsection{Requirements}
\begin{enumerate}
    \item {[Type here \textbf{DD1+}]}
    \item {[Type here \textbf{DD1+}]}
\end{enumerate}

\clearpage

\subsection{Factors Influencing Requirements}
\subsubsection{Public Health, Safety, and Welfare}
\begin{enumerate}
    \item {[Type here \textbf{DD1+}]}
    \item {[Type here \textbf{DD1+}]}
\end{enumerate}

\subsubsection{Global Factors}
\begin{enumerate}
    \item {[Type here \textbf{DD1+}]}
    \item {[Type here \textbf{DD1+}]}
\end{enumerate}

\subsubsection{Cultural Factors}
\begin{enumerate}
    \item Language independence. The device should
          be language-agnostic wherever possible.
    \item {[Type here \textbf{DD1+}]}
\end{enumerate}

\subsubsection{Social Factors}
\begin{enumerate}
    \item Accessibility. The device should be
          easily replicated with globally available parts.
    \item {[Type here \textbf{DD1+}]}
\end{enumerate}

\subsubsection{Environmental Factors}
\begin{enumerate}
    \item {[Type here \textbf{DD1+}]}
    \item {[Type here \textbf{DD1+}]}
\end{enumerate}

\subsubsection{Economic Factors}
\begin{enumerate}
    \item Affordability. The device must minimize
          construction costs and eliminate recurring
          costs (excepting power consumption).
    \item
\end{enumerate}
\clearpage
\section{System Overview}
\subsection{System Block Diagram}
\begin{figure}[h]
    \centering
    \includegraphics[width=16cm]{images/block_diagram.png}
    \caption{System Block Diagram}
\end{figure}

% Activity Diagram
\clearpage
\subsection{System Activity Diagram}
[\textbf{DD1+}]
\begin{figure}[h]
    \centering
    \includegraphics[width=16cm]{images/white.png} % Change the picture
    \caption{System Activity Diagram}
\end{figure}

% Mechanical Design
\clearpage
\subsection{System Mechanical Design (Extra Credit)}
[\textbf{DD3+}]
\begin{figure}[h]
    \centering
    \includegraphics[width=16cm]{images/white.png} % Change the picture
    \caption{System Mechanical Design}
\end{figure}

% Integration Approach
\clearpage
\subsection{Integration Approach}
[\textbf{DD3+}]
[Theory behind the system design, with reference to subsystem integration within your system – i.e., explain how it is supposed to work, but not whether it did actually work]
[Type here]

% System Photograph
\clearpage
\subsection{System Photographs} % Have as many photoes as you need
[\textbf{DD3+}]
[Photograph of assembled system, intended to highlight user interaction / controls. If system is split into multiple parts, show a composite of more than one photograph with all key user interactions / controls. ]
\begin{figure}[h]
    \centering
    \includegraphics[width=16cm]{images/white.png} % Change the picture
    \caption{[Photo Name]}
\end{figure}
\section{Subsystems}
\subsection{Subsystem 1: Processing}

\subsubsection{Subsystem Diagrams}
\begin{figure}[h]
    \centering
    \includegraphics[width=12cm]{images/Processing_Subsystem_Block_Diagram.png} % Change the picture
    \caption{Processing subsystem Block Diagram}
\end{figure} % If your subsystem is more coding, change it to activity diagram

% Specifications
\subsubsection{Specifications}
\begin{enumerate}
    \item Support SPI clock $\geq$ 20 MHz
    \item SPI support up to 40 MHz, full-duplex, DMA capable
    \item Display update latency $\leq$ 16 ms
    \item Support configurable API refresh interval between 60 and 300 s
    \item API end-to-end fetch latency $\leq$ 2 s
    \item Clock drift $\leq \pm 1\frac{\text{sec}}{\text{day}}$
    \item Operating voltage: $3.3 \pm 0.1$ V
\end{enumerate}

% Subsystem Interactions
\subsubsection{Subsystem Interactions}
The core computer interfaces with
all other subsystems. The battery/power
management unit supplies it with power.
Running processes direct and receive
information from the network module.
It communicates with the digital dot matrix
display via SPI according to display
drivers on the controller.


% Core ECE
\subsubsection{Core ECE Design Tasks}
\begin{itemize}
    \item \textbf{ENGR 16100}: Teamwork \& project documentation.
    \item \textbf{CS 15900}: Fundamentals of programming.
    \item \textbf{ECE 36200}: PCB design and embedded software development.
\end{itemize}

% Parts List
\subsubsection{Parts}
\begin{itemize}
    \item ESP32-S3
    \item DS3231
    \item PCB
\end{itemize}

% Algorithm
\subsubsection{Algorithm}
\begin{lstlisting}
    initialize clock, network, display, location
    DATA = {}

    always 
        wifi keep alive
        error handling
        if power button pressed
            save and shut down
    
    every minute
        update display time

        read ADS-B sensor
        if ADS-B valid
            add to DATA
            if internet connected
                upload ADS-B data
            else
                cache ADS-B data
        else 
            if internet connected
                make API call for flight data
                add to DATA
            else
                add warning ADS-B out of date to DATA
            

        if battery powered
            update battery level in DATA

        update time, weather in DATA
        convert DATA to pixel buffer
        push pixel buffer to displar
    every hour
        sync with network time
        make API call for weather

\end{lstlisting}

\subsubsection{Theory of Operation}
The processing subsystem is responsible for
coordinating all other subsystems and serving
as the “brain” of the Aviator device. At startup,
the ESP32-S3 initializes its real-time clock,
connects to the Wi-Fi network, and
configures the SPI interface for display communication.
During normal operation, the processor maintains a
persistent network connection to ensure timely access
to APIs.

Every minute, the processor updates the displayed time,
retrieves the latest flight information from the network
subsystem, and parses the response into a standardized
data structure. If the device is battery-powered, it
also queries the power management subsystem for voltage
information and appends this to the displayed data.
The processor then converts the compiled information
(time, weather, and flight updates) into a pixel buffer
and transmits it to the display subsystem.

On an hourly basis, the processor synchronizes the
real-time clock with network time servers and fetches
weather data for local context. Error handling
routines ensure that loss of network, corrupted data,
or power fluctuations do not crash the system; instead,
the subsystem retries API calls or falls back to cached
data, with a warning that the data is out of date.

% Specification Measurement
\subsubsection{Specifications Measurement}
[\textbf{DD3+} Every specification here should match the specification above. ]
\begin{enumerate}
    \item {[Copy specification here. ]} \\
          {[Explain the specification here. Add photoes if necessary. ]}
\end{enumerate}

% Standards
\subsubsection{Standards}
\begin{itemize}
    \item \textbf{IPC-2221}: Governs PCB trace width, spacing, creepage/clearance, via rules, grounding, etc.
    \item \textbf{IPC-A-610}: Covers soldering quality and workmanship.
    \item \textbf{RFC 5905}: Protocol for syncing ESP time to internet.
\end{itemize}
\clearpage
\subsection{Subsystem 2: Power Management}

% Subsystem Diagram
\subsubsection{Subsystem Diagrams}
\begin{figure}[h]
    \centering
    \includegraphics[width=16cm]{images/Power/BlockDiagram-power.jpg} % Change the picture
    \caption{Power subsystem Block Diagram}
\end{figure} % If your subsystem is more coding, change it to activity diagram

% Specifications
\subsubsection{Specifications}
\begin{enumerate}
    \item {Input Voltage: 5 V from USB-C adapter.}
    \item {Backup Source: 3.7 V Li-ion battery (at least 8000 mAh)}
    \item {Output Rails:}
    \item {5 V for Display (load up to 3 A)}
    \item {3.3 V for ESP32 (load up to 1 A)}
    \item {Voltage Accuracy: ± 5 \%}
    \item {Average Power Consumption: < 8 W}
    \item {Efficiency: 85\% for buck conversion; 80\% charge/discharge.}
    \item {Protection: fuse (3 A hold)}
    \item {Thermal Limit: surface temperature under 60 °C under full load.}
\end{enumerate}

% Subsystem Interactions
\subsubsection{Subsystem Interactions}
\begin{enumerate}
    \item {With Processing Subsystem (Zeke): Provides regulated 3.3 V rail to ESP32}
    \item {With Display Subsystem (David): Delivers 5 V rail and handles power demand spikes during high-brightness operation.}
    \item {With PCB Integration (Zeke and Joshua): Power traces routed centrally from power management module on PCB, decoupling capacitors placed near load headers to reduce noise}
    \item {With Networking Subsystem (Juan): Shares 3.3 V supply for network module and provides stable voltage during transmit peaks.}
\end{enumerate}

% Core ECE
\subsubsection{Core ECE Design Tasks}
\begin{itemize}
    \item \textbf{ECE 25500}: Electronic Devices and Design Lab: Used to design buck-boost converter and protection circuits
    \item \textbf{ECE 27000}: Digital Systems: Interfaces battery and power-good signals with ESP32 logic pins
    \item \textbf{ECE 36900}: Power Electronics: Guides design of buck and buck-boost stages and load sharing
    \item \textbf{ECE 49595}: Embedded Hardware Design: Applied to joint PCB work with Zeke
\end{itemize}

% Schematics
\subsubsection{Schematics}
[Type here \textbf{DD2+}]
\begin{figure}[h]
    \centering
    \includegraphics[width=16cm]{images/white.png} % Change the picture
    \caption{[Schematic Name]}
\end{figure} % If your subsystem is more coding, change it to psudo code

% Parts List
\subsubsection{Parts}
\begin{itemize}
    \item USB-C 5 V input module
    \item Li-ion battery pack (3.7 V, 8000 mAh)
    \item Buck-boost regulator IC (for 3.3 V and 5 V rails)
    \item Battery charging IC (USB-C PD or simple Li-ion charger).
    \item Basic protection devices (fuse, MOSFET switch, TVS diode).
    \item (Optional) Solar panel (5–20 W) + MPPT controller IC.
\end{itemize}
\subsubsection{Finalized Parts}
\begin{itemize}
    \item USB-C input controller: STUSB4500 (PD sink, 5 V mode)
    \item Buck-boost regulator: TI TPS63070 (3.3 V / 5 V output from battery or adapter)
    \item Battery charging IC: TP4056 (simple Li-ion charger for 5 V input)
    \item Li-ion battery pack: 3.7 V 8000 mAh flat pack with PCM protection
    \item Fuse: resettable polyfuse (3 A hold)
    \item TVS diode: SMBJ5.0A (for transient suppression)
    \item MOSFET switch: P-channel AO3407A (for reverse protection and load control)
    \item Connectors / Headers: keyed 5 V and 3.3 V output headers with test pads
\end{itemize}
% Algorithm
\subsubsection{Algorithm}
\begin{itemize}
    \item Initialization: detect input source (USB-C present or battery only).
    \item If USB-C is present: supply system load directly and charge battery at regulated current.
    \item If USB-C is removed: automatically switch to battery through PowerPath MOSFET (ideal-diode behavior).
    \item Monitor: ESP32 reads battery voltage and power-good signal periodically.
    \item Low-voltage event: ESP32 alerts display and reduces brightness to conserve power.
\end{itemize}

% Theory of Operation
\subsubsection{Theory of Operation}
The power management subsystem accepts a 5 V input from a USB-C adapter as its primary source. 
When connected, the charger IC diverts current both to the system load and to the battery for charging. 
A buck-boost converter regulates voltage for both 5 V and 3.3 V rails regardless of whether the system is running on the adapter or battery. 
If the adapter is unplugged, the PowerPath MOSFET automatically connects the battery to the load without noticeable interruption (< 100 ms). 
Protection devices (fuse and TVS diode) guard against short circuits and voltage spikes. 
The ESP32 monitors the battery status and can display charging or battery mode indicators.
% Specification Measurement
\subsubsection{Specifications Measurement}
[\textbf{DD3+} Every specification here should match the specification above. ]
\begin{enumerate}
    \item {[Copy specification here. ]} \\
          {[Explain the specification here. Add photoes if necessary. ]}
\end{enumerate}

% Standards
\subsubsection{Standards}
\begin{itemize}
    \item \textbf{USB-C Power Delivery Specification}: ensures safe and standard input power.
    \item \textbf{IEEE 1625}: covers battery system reliability for portable electronics.
    \item \textbf{UL 2054}: safety standard for rechargeable batteries in consumer devices.
    \item \textbf{IEC 61215}: (Optional solar) performance and reliability for PV modules.
\end{itemize}

\subsection{Subsystem 3: Text Display \& Chassis}

% Subsystem Diagram
\subsubsection{Subsystem Diagrams}
\begin{figure}[h]
    \centering
    \includegraphics[width=12cm]{images/blockDiagram-display.png} % Change the picture
    \caption{Subsystem Block Diagram}
\end{figure} % If your subsystem is more coding, change it to activity diagram

% Specifications
\subsubsection{Specifications}
\begin{enumerate}
    \item Feature a 2056-pixel Dot Display (128 x 256px)
    \item Physical dimensions of housing within 5\% of 64 x 128 x 256 mm.
\end{enumerate}

% Subsystem Interactions
\subsubsection{Subsystem Interactions}
The display subsystem links the main PCB and the LED Matrix. This subsystem will interface with the power management system and the main PCB. It will recieve data and commands from the main PCB, interpret the commands, and display the data.

% Core ECE
\subsubsection{Core ECE Design Tasks}
\begin{itemize}
    \item \textbf{ECE 27000}: Provides a solid foundation in logic circuits. Helps for designing registers, data paths, logic that drives the LED panel.
    \item \textbf{ECE 25500}: Provides a base understanding of transisters, amplifiers, and fundamentals into I-V behavior.
    \item \textbf{ECE 40862}: Teaches valuable STM concepts such as programming, interrupts, and DMA which are needed to drive the display controller efficiently.
\end{itemize}

% Schematics
\subsubsection{Schematics}
[Type here \textbf{DD2+}]
\begin{figure}[h]
    \centering
    \includegraphics[width=16cm]{images/white.png} % Change the picture
    \caption{[Schematic Name]}
\end{figure} % If your subsystem is more coding, change it to psudo code

% Parts List
\subsubsection{Parts}
\begin{itemize}
    \item 32 x 64 px LED Matrix (ADAFruit from DigiKey)
    \item Custom PCB (possibly a hat or extension) for driving display
\end{itemize}

% Algorithm
\subsubsection{Algorithm}

\begin{lstlisting}
    Initialize:
    Configure GPIOs (DATA, CLK, LAT, OE, row select lines (A,B,C,D))
    Initialize interface for incoming data
    Initialize timer for refresh interrupts
    Set PWM resolution (e.g., 8-bit)

    Main Loop:
    while (true):
        if new frame data available from main PCB:
            copy frame buffer into local memory (double buffer optional)
        
        for each row_index in 0 .. NUM_ROWS-1:
            select_row(row_index)        
            OE = HIGH                   
            
            for pwm_bit = 7 downto 0:    // for 8-bit PWM
                for col_index = 0 .. NUM_COLS-1:
                    pixel = frame_buffer[row_index][col_index]
                    if pixel.red  & (1 << pwm_bit) != 0:
                        DATA_R = HIGH
                    else:
                        DATA_R = LOW
                    if pixel.green & (1 << pwm_bit) != 0:
                        DATA_G = HIGH
                    else:
                        DATA_G = LOW
                    if pixel.blue & (1 << pwm_bit) != 0:
                        DATA_B = HIGH
                    else:
                        DATA_B = LOW
                    
                    pulse(CLK)           
                    
                pulse(LAT)                
                OE = LOW                   
                delay(PWM_DELAY[pwm_bit])  
                
        repeat indefinitely
\end{lstlisting}

% Theory of Operation
\subsubsection{Theory of Operation}
[Type here \textbf{DD2+}]

% Specification Measurement
\subsubsection{Specifications Measurement}
[\textbf{DD3+} Every specification here should match the specification above. ]
\begin{enumerate}
    \item {[Copy specification here. ]} \\
          {[Explain the specification here. Add photoes if necessary. ]}
\end{enumerate}

% Standards
\subsubsection{Standards}
\begin{itemize}
    \item \textbf{IEC 61010}: Safety for low-voltage electronic equipment; ensures protection against shorts, overcurrent, and handling risks.
    \item \textbf{SPI/I²C}: If MCU/controller communicates with peripheral ICs over standard buses.
    \item \textbf{HUB75}: Standard for 32×64 RGB LED matrices; timing, row multiplexing, and data latching must be followed.
    \item \textbf{JESD51}: Important for high-current LED arrays; ensures heat dissipation and junction temperatures are within safe limits.
\end{itemize}
%%%%%%%%%%%%% Subsystem 4 %%%%%%%%%%%%%%%
\clearpage
\subsection{Subsystem 4: [Subsystem Name]}

% Subsystem Diagram
\subsubsection{Subsystem Diagrams}
[\textbf{DD1+}]
\begin{figure}[h]
    \centering
    \includegraphics[width=16cm]{images/white.png} % Change the picture
    \caption{Subsystem Block Diagram}
\end{figure} % If your subsystem is more coding, change it to activity diagram

% Specifications
\subsubsection{Specifications}
\begin{enumerate}
    \item {[Type here \textbf{DD1+}]}
\end{enumerate}

% Subsystem Interactions
\subsubsection{Subsystem Interactions}
[Type here \textbf{DD1+}]

% Core ECE
\subsubsection{Core ECE Design Tasks}
[\textbf{DD1+} Write tasks and course that helps accomplish that task]
\begin{itemize}
    \item \textbf{ECE xxxxx}: [Type the relationship here. ]
\end{itemize}

% Schematics
\subsubsection{Schematics}
[Type here \textbf{DD2+}]
\begin{figure}[h]
    \centering
    \includegraphics[width=16cm]{images/white.png} % Change the picture
    \caption{[Schematic Name]}
\end{figure} % If your subsystem is more coding, change it to psudo code

% Parts List
\subsubsection{Parts}
\begin{itemize}
    \item {[Type here \textbf{DD1+}]}
\end{itemize}

% Algorithm
\subsubsection{Algorithm}
[Type here \textbf{DD1+}]

% Theory of Operation
\subsubsection{Theory of Operation}
[Type here \textbf{DD2+}]

% Specification Measurement
\subsubsection{Specifications Measurement}
[\textbf{DD3+} Every specification here should match the specification above. ]
\begin{enumerate}
    \item {[Copy specification here. ]} \\
          {[Explain the specification here. Add photoes if necessary. ]}
\end{enumerate}

% Standards
\subsubsection{Standards}
[\textbf{DD1+}]
\begin{itemize}
    \item \textbf{[Standard Name]}: [Describe the standards and explain the connection]
\end{itemize}
\clearpage
\section{PCB Design}
\subsection{PCB Schematics}
[\textbf{DD3+}]
\begin{figure}[!h]
    \centering
    \includegraphics[width=10cm]{images/white.png} % Change the picture
    \caption{PCB Schematic}
\end{figure}

\clearpage
\subsection{PCB Layout}
[\textbf{DD3+}]
\begin{figure}[h]
    \centering
    \includegraphics[width=16cm]{images/white.png} % Change the picture
    \caption{PCB Layout}
\end{figure}
%%%%%%%%%%%%%%% Final Status Page %%%%%%%%%%%%%%%%%%
\clearpage
\section{Final Status of Requirements}
 [\textbf{DD3+}]
 [If met, give a detailed explanation of the requirement. If partially met, mention what has been met and a reason for why the complete requirement couldn’t be achieved. If not met, give an explanation for why the requirement couldn’t be met in the product. Add as many requirements as you had in your earlier design documents here. ]
\begin{enumerate}
    \item Requirement 1: [Copy your requirement above here] \\
          \textbf{Met}: [Explanation]
    \item Requirement 2: [Copy your requirement above here] \\
          \textbf{Partially Met}: [Explanation]
    \item Requirement 3: [Copy your requirement above here] \\
          \textbf{Not Met}: [Explanation]
\end{enumerate}

%%%%%%%%%%%%%%% Team Intro Page %%%%%%%%%%%%%%%%%%%%
\clearpage
\section{Team Structure}
 [\textbf{DD1+}]
\subsection{Team Member 1}
\includegraphics[height=4cm]{images/white.png} \\
\textbf{[Name Here]}\\
Major: [FILL IN]\\
Contact: [user]@purdue.edu\\
Team Role: [Technical and Professional Roles in the team] \\
Bio: [Short Introduction here]

\subsection{Team Member 2}
\includegraphics[height=4cm]{images/white.png} \\
\textbf{[Name Here]}\\
Major: [FILL IN]\\
Contact: [user]@purdue.edu\\
Team Role: [Technical and Professional Roles in the team] \\
Bio: [Short Introduction here]

\subsection{Team Member 3}
\includegraphics[height=4cm]{images/zeke.png} \\
\textbf{Zeke Ulrich}\\
Major: Computer Engineering\\
Contact: pulrich@purdue.edu\\
Team Role: Treasurer \\
Bio: Zeke is the processing and PCB design specialist.
At Purdue, he belongs to the Marine Corp Officer Candidate Program,
Eta Kappa Nu, Tau Beta Pi, and Purdue's Effective Altruism community.
Outside Purdue, he is president of the nonprofit DuelGood and works
for the government in cybersecurity.
In the future he hopes to study international relations as a Truman
scholar, start a family, and volunteer as a firefighter.
He enjoys athletics and spending time with his friends.

\subsection{Team Member 4}
\includegraphics[height=4cm]{images/white.png} \\
\textbf{[Name Here]}\\
Major: [FILL IN]\\
Contact: [user]@purdue.edu\\
Team Role: [Technical and Professional Roles in the team] \\
Bio: [Short Introduction here]

%%%%%%%%%%%%%%%%% Biblography %%%%%%%%%%%%%%%%%%%%%%%
\clearpage
\section{Bibliography}

 [Here are some examples. IEEE format can be found on \href{https://owl.purdue.edu/owl/research_and_citation/ieee_style/ieee_overview.html}{\hl{Purdue OWL}}. ]

\begin{thebibliography}{}

    \bibitem{b1}
    “Data Platform - Open Power System data,” Apr. 15, 2020. https://data.open-power-system-data.org/household\_data/

    \bibitem{b2}
    Author,"Title," \emph{Journal},volume,number, page range, month year, DOI.

    \bibitem{b3}
    Author. "Page."Website. URL(accessed month day,year)

\end{thebibliography}

%%%%%%%%%%%%%%%%% Appendix %%%%%%%%%%%%%%%%%%%%%%%%%
\clearpage
\section{Appendices}
 [This section is mainly designed for code.
  You can directly generate a somewhat decent display
  of your code file or psuedo code by using the template provided below. You can have as many appendix as you want. In the document, you can refer to the code posted here instead of pasting the whole code in the body. ]

\include{chapters/appendix}

\end{document}
