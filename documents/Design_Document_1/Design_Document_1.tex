\documentclass[letterpaper, 11pt]{article}
\usepackage{cite,enumitem,listings, comment}
\usepackage[export]{adjustbox}
\usepackage{amsmath,amssymb,amsfonts}
\usepackage{algorithmic}
\usepackage{graphicx}
\usepackage{textcomp}
\usepackage{graphicx}
\usepackage{geometry}
\usepackage{tabularx}
\usepackage{xcolor}
\usepackage{soul}
\usepackage{chngcntr}
\usepackage{url}
\usepackage{hyperref}
\counterwithin{figure}{subsection}

\begin{document}

%%%%%%%%%%%%%%%% Document Starts %%%%%%%%%%%%%%%%%%%%%

% Title page 
\title{[Project Name]}
\author{[Team Member], [Team Member], [Team Member], [Team Member]}
\date{[Date]}
\maketitle
\begin{center}
    GTA: [GTA Name] \\
    Professor: [Professor Name]
\end{center}
\begin{center}
    \textbf{Design Document}
\end{center}

% Sketch goes here
\begin{figure}[h]
    \centering
    \includegraphics[width=12cm,scale=1]{white.png}
    \caption{[Caption]}
\end{figure}

%%%%%%%%%%%%%%%% Contents pages %%%%%%%%%%%%%%%%%%%%%%%

% In these pages, if you are in Overleaf, you can click on the content to go to the section. 
\clearpage
\tableofcontents % No need to edit anything. It should update automatically. 

% List of table and figures
\clearpage
\listoffigures % No need to edit anything. It should update automatically. 

\clearpage
\listoftables % No need to edit anything. It should update automatically. 

% Revision Log Page
\clearpage
\section*{Revision Log}
% For every update, copy a whole LATEX code line and adjust accordingly. In theory, update every time you make major changes in the document. The \& symbol serves to separate items in the table. If you are new to this , aligh the \& so that it looks more like a table in the code. 
\begin{table}[h]
    \begin{tabularx}{\textwidth}{|l|c|X|} % 
        \hline
        Date     & Revision & Changes         \\ \hline
        5/3/2024 & v0.1     & Initial Release \\ \hline
        [Copy]   & [and]    & [Replace]       \\ \hline
    \end{tabularx}
    \caption{Revision Log}
\end{table}

% Glossary Page
\clearpage
\section*{Glossary} % Document key words and acronyms here. 
\begin{itemize} % Remember to bold format the key word. Here are some examples. 
    \item \textbf{3D audio technology} Simulation that creates the illusion of sound sources placed anywhere in 3 dimensional space, including behind, above or below the listener.
    \item \textbf{API} Application Programming Interface.
    \item
\end{itemize}

% Introduction Page
\clearpage
\section{Introduction}
\subsection{Executive Description}
[Type here. \textbf{DD1+}]
\subsection{User Story}
[Type here. \textbf{DD1+}]

%%%%%%%%%%%%%%%%%%%%%%%% Design Requirements Pages %%%%%%%%%%%%%%%%%%%%%%%%%%%
\clearpage

% Requirements
\section{Design Requirements}
\subsection{Requirements}
\begin{enumerate}
    \item {[Type here \textbf{DD1+}]}
    \item {[Type here \textbf{DD1+}]}
\end{enumerate}

\clearpage

\subsection{Factors influencing requirements}
\subsubsection{Public Health, Safety, and Welfare}
\begin{enumerate}
    \item {[Type here \textbf{DD1+}]}
    \item {[Type here \textbf{DD1+}]}
\end{enumerate}

\subsubsection{Global Factors}
\begin{enumerate}
    \item {[Type here \textbf{DD1+}]}
    \item {[Type here \textbf{DD1+}]}
\end{enumerate}

\subsubsection{Cultural Factors}
\begin{enumerate}
    \item {[Type here \textbf{DD1+}]}
    \item {[Type here \textbf{DD1+}]}
\end{enumerate}

\subsubsection{Social Factors}
\begin{enumerate}
    \item {[Type here \textbf{DD1+}]}
    \item {[Type here \textbf{DD1+}]}
\end{enumerate}

\subsubsection{Environmental Factors}
\begin{enumerate}
    \item {[Type here \textbf{DD1+}]}
    \item {[Type here \textbf{DD1+}]}
\end{enumerate}

\subsubsection{Economical Factors}
\begin{enumerate}
    \item {[Type here \textbf{DD1+}]}
    \item {[Type here \textbf{DD1+}]}
\end{enumerate}

\clearpage

%%%%%%%%%%%%%%% System Overview Pages %%%%%%%%%%%%%%%%%%%%
\section{System Overview}
% Block Diagram
\subsection{System Block Diagram}
[\textbf{DD1+}]
\begin{figure}[h]
    \centering
    \includegraphics[width=16cm]{white.png} % Change the picture
    \caption{System Block Diagram}
\end{figure}

% Activity Diagram
\clearpage
\subsection{System Activity Diagram}
[\textbf{DD1+}]
\begin{figure}[h]
    \centering
    \includegraphics[width=16cm]{white.png} % Change the picture
    \caption{System Activity Diagram}
\end{figure}

% Mechanical Design
\clearpage
\subsection{System Mechanical Design (Extra Credit)}
[\textbf{DD3+}]
\begin{figure}[h]
    \centering
    \includegraphics[width=16cm]{white.png} % Change the picture
    \caption{System Mechanical Design}
\end{figure}

% Integration Approach
\clearpage
\subsection{Integration Approach}
[\textbf{DD3+}]
[Theory behind the system design, with reference to subsystem integration within your system – i.e., explain how it is supposed to work, but not whether it did actually work]
[Type here]

% System Photograph
\clearpage
\subsection{System Photographs} % Have as many photoes as you need
[\textbf{DD3+}]
[Photograph of assembled system, intended to highlight user interaction / controls. If system is split into multiple parts, show a composite of more than one photograph with all key user interactions / controls. ]
\begin{figure}[h]
    \centering
    \includegraphics[width=16cm]{white.png} % Change the picture
    \caption{[Photo Name]}
\end{figure}

%%%%%%%%%%%%%%%% Subsystem Page %%%%%%%%%%%%%%%%%%
\clearpage
\section{Subsystems}

%%%%%%%%%%%%% Subsystem 1 %%%%%%%%%%%%%%%
\subsection{Subsystem 1: [Subsystem Name]}

% Subsystem Diagram
\subsubsection{Subsystem Diagrams}
[\textbf{DD1+}]
\begin{figure}[h]
    \centering
    \includegraphics[width=16cm]{white.png} % Change the picture
    \caption{Subsystem Block Diagram}
\end{figure} % If your subsystem is more coding, change it to activity diagram

% Specifications
\subsubsection{Specifications}
\begin{enumerate}
    \item {[Type here \textbf{DD1+}]}
\end{enumerate}

% Subsystem Interactions
\subsubsection{Subsystem Interactions}
[Type here \textbf{DD1+}]

% Core ECE
\subsubsection{Core ECE Design Tasks}
[\textbf{DD1+} Write tasks and course that helps accomplish that task]
\begin{itemize}
    \item \textbf{ECE xxxxx}: [Type the relationship here. ]
\end{itemize}

% Schematics
\subsubsection{Schematics}
[Type here \textbf{DD2+}]
\begin{figure}[h]
    \centering
    \includegraphics[width=16cm]{white.png} % Change the picture
    \caption{[Schematic Name]}
\end{figure} % If your subsystem is more coding, change it to psudo code

% Parts List
\subsubsection{Parts}
\begin{itemize}
    \item {[Type here \textbf{DD1+}]}
\end{itemize}

% Algorithm
\subsubsection{Algorithm}
[Type here \textbf{DD1+}]

% Theory of Operation
\subsubsection{Theory of Operation}
[Type here \textbf{DD2+}]

% Specification Measurement
\subsubsection{Specifications Measurement}
[\textbf{DD3+} Every specification here should match the specification above. ]
\begin{enumerate}
    \item {[Copy specification here. ]} \\
          {[Explain the specification here. Add photoes if necessary. ]}
\end{enumerate}

% Standards
\subsubsection{Standards}
[\textbf{DD1+}]
\begin{itemize}
    \item \textbf{[Standard Name]}: [Describe the standards and explain the connection]
\end{itemize}

%%%%%%%%%%%%% Subsystem 2 %%%%%%%%%%%%%%%
\clearpage
\subsection{Subsystem 2: [Subsystem Name]}

% Subsystem Diagram
\subsubsection{Subsystem Diagrams}
[\textbf{DD1+}]
\begin{figure}[h]
    \centering
    \includegraphics[width=16cm]{white.png} % Change the picture
    \caption{Subsystem Block Diagram}
\end{figure} % If your subsystem is more coding, change it to activity diagram

% Specifications
\subsubsection{Specifications}
\begin{enumerate}
    \item {[Type here \textbf{DD1+}]}
\end{enumerate}

% Subsystem Interactions
\subsubsection{Subsystem Interactions}
[Type here \textbf{DD1+}]

% Core ECE
\subsubsection{Core ECE Design Tasks}
[\textbf{DD1+} Write tasks and course that helps accomplish that task]
\begin{itemize}
    \item \textbf{ECE xxxxx}: [Type the relationship here. ]
\end{itemize}

% Schematics
\subsubsection{Schematics}
[Type here \textbf{DD2+}]
\begin{figure}[h]
    \centering
    \includegraphics[width=16cm]{white.png} % Change the picture
    \caption{[Schematic Name]}
\end{figure} % If your subsystem is more coding, change it to psudo code

% Parts List
\subsubsection{Parts}
\begin{itemize}
    \item {[Type here \textbf{DD1+}]}
\end{itemize}

% Algorithm
\subsubsection{Algorithm}
[Type here \textbf{DD1+}]

% Theory of Operation
\subsubsection{Theory of Operation}
[Type here \textbf{DD2+}]
% Specification Measurement
\subsubsection{Specifications Measurement}
[\textbf{DD3+} Every specification here should match the specification above. ]
\begin{enumerate}
    \item {[Copy specification here. ]} \\
          {[Explain the specification here. Add photoes if necessary. ]}
\end{enumerate}

% Standards
\subsubsection{Standards}
[\textbf{DD1+}]
\begin{itemize}
    \item \textbf{[Standard Name]}: [Describe the standards and explain the connection]
\end{itemize}

%%%%%%%%%%%%% Subsystem 3 %%%%%%%%%%%%%%%
\clearpage
\subsection{Subsystem 3: [Subsystem Name]}

% Subsystem Diagram
\subsubsection{Subsystem Diagrams}
[\textbf{DD1+}]
\begin{figure}[h]
    \centering
    \includegraphics[width=16cm]{white.png} % Change the picture
    \caption{Subsystem Block Diagram}
\end{figure} % If your subsystem is more coding, change it to activity diagram

% Specifications
\subsubsection{Specifications}
\begin{enumerate}
    \item {[Type here \textbf{DD1+}]}
\end{enumerate}

% Subsystem Interactions
\subsubsection{Subsystem Interactions}
[Type here \textbf{DD1+}]

% Core ECE
\subsubsection{Core ECE Design Tasks}
[\textbf{DD1+} Write tasks and course that helps accomplish that task]
\begin{itemize}
    \item \textbf{ECE xxxxx}: [Type the relationship here. ]
\end{itemize}

% Schematics
\subsubsection{Schematics}
[Type here \textbf{DD2+}]
\begin{figure}[h]
    \centering
    \includegraphics[width=16cm]{white.png} % Change the picture
    \caption{[Schematic Name]}
\end{figure} % If your subsystem is more coding, change it to psudo code

% Parts List
\subsubsection{Parts}
\begin{itemize}
    \item {[Type here \textbf{DD1+}]}
\end{itemize}

% Algorithm
\subsubsection{Algorithm}
[Type here \textbf{DD1+}]

% Theory of Operation
\subsubsection{Theory of Operation}
[Type here \textbf{DD2+}]

% Specification Measurement
\subsubsection{Specifications Measurement}
[\textbf{DD3+} Every specification here should match the specification above. ]
\begin{enumerate}
    \item {[Copy specification here. ]} \\
          {[Explain the specification here. Add photoes if necessary. ]}
\end{enumerate}

% Standards
\subsubsection{Standards}
[\textbf{DD1+}]
\begin{itemize}
    \item \textbf{[Standard Name]}: [Describe the standards and explain the connection]
\end{itemize}

%%%%%%%%%%%%% Subsystem 4 %%%%%%%%%%%%%%%
\clearpage
\subsection{Subsystem 4: [Subsystem Name]}

% Subsystem Diagram
\subsubsection{Subsystem Diagrams}
[\textbf{DD1+}]
\begin{figure}[h]
    \centering
    \includegraphics[width=16cm]{white.png} % Change the picture
    \caption{Subsystem Block Diagram}
\end{figure} % If your subsystem is more coding, change it to activity diagram

% Specifications
\subsubsection{Specifications}
\begin{enumerate}
    \item {[Type here \textbf{DD1+}]}
\end{enumerate}

% Subsystem Interactions
\subsubsection{Subsystem Interactions}
[Type here \textbf{DD1+}]

% Core ECE
\subsubsection{Core ECE Design Tasks}
[\textbf{DD1+} Write tasks and course that helps accomplish that task]
\begin{itemize}
    \item \textbf{ECE xxxxx}: [Type the relationship here. ]
\end{itemize}

% Schematics
\subsubsection{Schematics}
[Type here \textbf{DD2+}]
\begin{figure}[h]
    \centering
    \includegraphics[width=16cm]{white.png} % Change the picture
    \caption{[Schematic Name]}
\end{figure} % If your subsystem is more coding, change it to psudo code

% Parts List
\subsubsection{Parts}
\begin{itemize}
    \item {[Type here \textbf{DD1+}]}
\end{itemize}

% Algorithm
\subsubsection{Algorithm}
[Type here \textbf{DD1+}]

% Theory of Operation
\subsubsection{Theory of Operation}
[Type here \textbf{DD2+}]

% Specification Measurement
\subsubsection{Specifications Measurement}
[\textbf{DD3+} Every specification here should match the specification above. ]
\begin{enumerate}
    \item {[Copy specification here. ]} \\
          {[Explain the specification here. Add photoes if necessary. ]}
\end{enumerate}

% Standards
\subsubsection{Standards}
[\textbf{DD1+}]
\begin{itemize}
    \item \textbf{[Standard Name]}: [Describe the standards and explain the connection]
\end{itemize}

%%%%%%%%%%%%%%% PCB Page %%%%%%%%%%%%%%%%%
\clearpage
\section{PCB Design}
\subsection{PCB Schematics}
[\textbf{DD3+}]
\begin{figure}[!h]
    \centering
    \includegraphics[width=10cm]{white.png} % Change the picture
    \caption{PCB Schematic}
\end{figure}

\clearpage
\subsection{PCB Layout}
[\textbf{DD3+}]
\begin{figure}[h]
    \centering
    \includegraphics[width=16cm]{white.png} % Change the picture
    \caption{PCB Layout}
\end{figure}

%%%%%%%%%%%%%%% Final Status Page %%%%%%%%%%%%%%%%%%
\clearpage
\section{Final Status of Requirements}
 [\textbf{DD3+}]
 [If met, give a detailed explanation of the requirement. If partially met, mention what has been met and a reason for why the complete requirement couldn’t be achieved. If not met, give an explanation for why the requirement couldn’t be met in the product. Add as many requirements as you had in your earlier design documents here. ]
\begin{enumerate}
    \item Requirement 1: [Copy your requirement above here] \\
          \textbf{Met}: [Explanation]
    \item Requirement 2: [Copy your requirement above here] \\
          \textbf{Partially Met}: [Explanation]
    \item Requirement 3: [Copy your requirement above here] \\
          \textbf{Not Met}: [Explanation]
\end{enumerate}

%%%%%%%%%%%%%%% Team Intro Page %%%%%%%%%%%%%%%%%%%%
\clearpage
\section{Team Structure}
 [\textbf{DD1+}]
\subsection{Team Member 1}
\includegraphics[height=4cm]{white.png} \\
\textbf{[Name Here]}\\
Major: [FILL IN]\\
Contact: [user]@purdue.edu\\
Team Role: [Technical and Professional Roles in the team] \\
Bio: [Short Introduction here]

\subsection{Team Member 2}
\includegraphics[height=4cm]{white.png} \\
\textbf{[Name Here]}\\
Major: [FILL IN]\\
Contact: [user]@purdue.edu\\
Team Role: [Technical and Professional Roles in the team] \\
Bio: [Short Introduction here]

\subsection{Team Member 3}
\includegraphics[height=4cm]{white.png} \\
\textbf{[Name Here]}\\
Major: [FILL IN]\\
Contact: [user]@purdue.edu\\
Team Role: [Technical and Professional Roles in the team] \\
Bio: [Short Introduction here]

\subsection{Team Member 4}
\includegraphics[height=4cm]{white.png} \\
\textbf{[Name Here]}\\
Major: [FILL IN]\\
Contact: [user]@purdue.edu\\
Team Role: [Technical and Professional Roles in the team] \\
Bio: [Short Introduction here]

%%%%%%%%%%%%%%%%% Biblography %%%%%%%%%%%%%%%%%%%%%%%
\clearpage
\section{Bibliography}

 [Here are some examples. IEEE format can be found on \href{https://owl.purdue.edu/owl/research_and_citation/ieee_style/ieee_overview.html}{\hl{Purdue OWL}}. ]

\begin{thebibliography}{}

    \bibitem{b1}
    “Data Platform - Open Power System data,” Apr. 15, 2020. https://data.open-power-system-data.org/household\_data/

    \bibitem{b2}
    Author,"Title," \emph{Journal},volume,number, page range, month year, DOI.

    \bibitem{b3}
    Author. "Page."Website. URL(accessed month day,year)

\end{thebibliography}

%%%%%%%%%%%%%%%%% Appendix %%%%%%%%%%%%%%%%%%%%%%%%%
\clearpage
\section{Appendices}
 [This section is mainly designed for code. You can directly generate a somewhat decent display of your code file or psuedo code by using the template provided below. You can have as many appendix as you want. In the document, you can refer to the code posted here instead of pasting the whole code in the body. ]

\end{document}
